\chapter{บทนำ}
\label{ch:intro}
\section{ความสำคัญและที่มาของโครงงานศึกษาวิจัย }
ภาษีสุราเป็นการจัดเก็บรายได้ภาษีสรรพสามิตที่มีการจัดเก็บเป็นเงินงบประมาณเป็นส่วนหนึ่ง
ที่นำมาใช้ในการบริหารประเทศ   แต่ในปัจจุบันได้มีการลักลอบปลอมแปลงสุราชนิดต่าง ๆ  โดยเฉพาะสุราต่างประเทศเข้ามาขายในราคาที่ถูกเพราะไม่เสียภาษี ส่งผลกระทบให้สูญเสียรายได้ของรัฐจากการจัดเก็บรายได้ไป ผู้กระทำผิดได้มีรูปแบบในการปลอมแปลงแบบต่าง ๆ โดยหนึ่งในนั้นคือการปลอมแปลงแสตมป์ที่ใช้ปิดผนึกขวดสุรา ในปัจจุบันการตรวจว่าแสตมป์เป็นของจริงหรือของปลอมจะใช้บุคลากรของทางภาครัฐเป็นผู้ตรวจสอบ 
ทำให้ประสบกับปัญหาว่า บุคลลากรทางภาครัฐที่มีความชำนาญในการตรวจสอบแสตมป์มีจำนวนน้อย และเครื่องมือที่ใช้ในการตรวจสอบได้มาจากความชำนาญของบุคลลากร ผู้ซึ่งมีความเชี่ยวชาญในการตรวจสอบ  ทำให้การจับกุมเป็นไปด้วยความยากลำบาก  
 ด้วยเหตุดังกล่าวนี้โครงงานศึกษาวิจัยนี้จึงสนใจที่จะนำหลักการทางการประมวลผลภาพมาช่วยสร้างระบบอัตโนมัติสำหรับการตรวจสอบแสตมป์สุรา  

โครงงานศึกษาวิจัยนี้  ต้องการหาวิธีการตรวจสอบแสตมป์ภาษีสุรานำเข้าจากต่างประเทศว่าเป็นแสตมป์จริงหรือปลอม โดยใช้การถ่ายภาพด้วยกล้องดิจิทัลแบบพกพา ซึ่งได้มีงานวิจัยในอดีตที่มีแนวคิดของนำการประมวลผลภาพมาประยุกต์ใช้ในการงานการตรวจจับหาวัตถุ หรือตรวจสอบความถูกต้องอยู่จำนวนมาก ยกตัวอย่างงานของ Takeda~และคณะ~\cite{takeda1992expert} ตั้งแต่ปี ค.ศ. 1992 (พ.ศ.​ 2535) ได้นำเสนอแนวคิดระบบผู้เชี่ยวชาญ (expert system) มาใช้ในการตรวจสอบธนบัตร ในปีเดียวกัน Fukumi~และคณะ~\cite{fukumi1992rotation} ได้นำเสนอการรู้จำเหรียญในสภาวะมุมต่าง ๆ ด้วยการใช้เครือข่ายประสาทเทียมของคุณลักษณะเด่นของภาพ ในปี พ.ศ. 2541 อดิศร~ลีลาสันติธรรม~\cite{Adisorn41} ได้มีการนำการประมวลผลภาพมาใช้เพื่อการรู้จำและตรวจสอบธนบัตร  ซึ่งโดยปกติแล้ว นอกจากการสังเกตข้อมูลจากรูปภาพ, ลายภาพ, สี, จำนวนเงิน ฯลฯ ซึ่งเป็นคุณสมบัติภายนอกของแต่ละประเภทของธนบัตรแล้ว  ธนบัตรหลายประเภทมีการซ่อนภาพลายน้ำ  ซึ่งสามารถเห็นเป็นภาพเงา
เมื่อใช้แสงสว่างส่องที่ด้านหลังของธนบัตร  ในวิทยานิพนธ์ดังกล่าว ได้นำเสนอระบบการรู้จำธนบัตรอัตโนมัติ โดยใช้โครงข่ายประสาทเทียมแบบแพร่กลับและใช้ภาพภายนอกบนธนบัตรซึ่งเกิดจากการสะท้อนของแสงที่กระทบบนธนบัตรรวมกันกับภาพเงาของลายน้ำที่ซ่อนอยู่ซึ่งเกิดจากการส่องแสงสว่างด้านหลังธนบัตร  ในระบบนี้ขั้นแรกทำการหาขอบภาพสองระดับของธนบัตรที่เกิดจากการสะท้อนของแสงที่กระทบบนธนบัตร ขั้นที่สองทำการหาขอบภาพสองระดับของภาพเงาลายน้ำที่ซ่อนอยู่ 
ซึ่งเกิดจากการส่องแสงสว่างด้านหลังธนบัตร แล้วทำการรวมภาพกันในขั้นแรกและขั้นที่สอง ให้ได้ภาพเงาของลายน้ำบนบัตรพร้อมกับภาพธนบัตร จากนั้นทำการนอร์มอลไลซ์ภาพที่ได้ดังกล่าว          ให้มีมาตรฐานเดียวกัน และป้อนภาพดังกล่าวเข้าสู่โครงข่ายประสาทเทียมแบบแพร่กลับ 3 ชั้นเพื่อทำการเรียนรู้และการทดสอบ จากการทดสอบระบบที่นำเสนอข้างต้นเพื่อประเมินประสิทธิภาพของระบบ โดยใช้ตัวอย่างธนบัตรไทย 5 ชนิด  ( 20 บาท  50 บาท  100 บาท  500 บาท และ 1000 บาท ) ชนิดละ 80 ใบในการเรียนรู้ และทุกชนิด ๆ ละ 50 ใบเพื่อใช้ในการทดสอบ ผลการทดสอบโดยใช้โครงข่ายประสาทเทียมแบบแพร่กลับในการพิสูจน์ธนบัตรจริงและจำแนกชนิดของธนบัตร มีอัตราในการรู้จำได้ทั้งหมด  ในงานวิจัยของ Zhu~และคณะ~\cite{zhu2006robust} และงานของ Micenkov~และคณะ~\cite{micenkov2011stamp} ได้นำเสนอการตรวจจับการประดับตราในเอกสาร ซึ่งในงานทั้งสองชิ้นนี้มีลักษณะปัญหาการดึงเอาเฉพาะส่วนที่สนใจออกจากส่วนอื่น ๆ ในภาพ ซึ่งในปัญหาของการตรวจสอบแสตมป์ในโครงงานศึกษาวิจัยนั้นมีลักษณะทีีคล้ายกัน กล่าวคือส่วนสำคัญของแสตมป์สุราคือตรานกวายุภักษ์ จึงควรต้องมีการตัดเอาตราออกมาก่อน

แนวคิดสำคัญของการตัดเอาส่วนของภาพที่สนใจออกมาจากส่วนอื่นคือ การใช้รูปร่างของวัตถุที่ต้องการตัด เช่น วงกลม สี่เหลี่ยม เป็นต้น ในที่นี้เราจะสนใจเฉพาะการดึงส่วนที่เป็นวงกลมเนื่องจากตรานกวายุภักษ์ในแสตมป์สุรามีวงกลมล้อมรอบอยู่ วิธีการในการตรวจจับวงกลมที่นิยมใช้กันมากที่สุดคือการใช้การแปลงฮัฟ ซึ่งในงานของ Ioannou~และคณะ~\cite{ioannou1999circle} ได้นำเสนอแนวคิดของการใช้การแปลงฮัฟแบบ 2 มิติในการตรวจหาวงกลม การแปลงฮัฟทั่วไป (generalized hough transform) เป็นการแปลงภาพให้ไปอยู่ในโดเมนฮัฟ ซึ่งสามารถใช้การโหวตในการตรวจหาเส้นหรือส่วนโค้ง การแปลงฮัฟเพื่อการตรวจจับวงกลมเป็นกรณีพิเศษของการแปลงฮัฟทั่วไป แนวคิดอื่น ๆ ในการตรวจจับหาวงกลมได้แก่ งานของ Basalamah~\cite{basalamah2012histogram} ซึ่งนำเสนอการใช้ฮิสโตแกรมในการตรวจจับวงกลม

แม้ว่าจะมีงานวิจัยในอดีตจำนวนมากที่ได้นำเอาเทคนิคการประมวลผลภาพไปใช้แก้ปัญหาต่าง ๆ ก็ตาม ปัญหาการตรวจสอบความเป็นของจริงของแสตมป์มีลักษณะเฉพาะ และในความต้องการการใช้งานในลักษณะของการพกพา ในโครงงานศึกษาวิจัยนี้จึงต้องการค้นหาวิธีการที่มีความซับซ้อนน้อย แต่สามารถตรวจสอบแยกแยะแสตมป์จริงกับแสตมป์ปลอมได้

\section{วัตถุประสงค์ของโครงงานศึกษาวิจัย}
\begin{enumerate}
\item เพื่อพัฒนาและประยุกต์วิธีการประมวลผลภาพเชิงดิจิทัลในการคัดแยกแสตมป์จริงและปลอม
\item ทดสอบและหาวิธีการที่เหมาะสมในการคำนวณหาค่าเพื่อคัดแยก
%\item ระบบควบคุมการเลื่อนฉากอัตโนมัติ โดยรับข้อมูลระยะการเลื่อน
\end{enumerate}

\section{ขอบเขตของโครงงานศึกษาวิจัย}
\begin{enumerate}
\item คัดกรองแสตมป์รายได้ภาษีในวงกลม Hough แปลงการคัดแยกแสตมป์ด้วยการหาฮิสโตแกรม
\item การถ่ายภาพจะใช้กล้องตัวเดียวกันภายใต้สภาวะแวดล้อมคล้ายกัน
\item ใช้คอมพิวเตอร์ Window 7 64 bit core i7 CPU 2.30GHz ในการทดสอบ
\end{enumerate}
\newpage

\section{ประโยชน์จากโครงงานศึกษาวิจัย}
\begin{enumerate}
\item มีความรู้และความเข้าใจถึงหลักการของการประมวลผลภาพเชิงดิจิทัล
\item ทำให้ได้วิธีในการคัดแยกแสตมป์จริงและแสตมป์ปลอมที่ช่วยให้การประมวนผลทำได้สะดวกรวดเร็วและมีความแม่นยำสูง
%\item ได้ระบบที่สามารถนำไปประยุกต์ใช้งานจัดท่าถ่ายภาพเอกซ์เรย์เพื่อใช้ในการตรวจสุขภาพ ทั้งที่โรงพยาบาลและรถเอกซ์เรย์เคลื่อนที่
\end{enumerate}

\section{ระเบียบวิธีการดำเนินวิจัย}
งานวิจัยในโครงการวิจัยนี้เป็นงานวิจัยเชิงวิศวกรรมเพื่อค้นหาวิธีการทางการประมวลผลภาพที่เหมาะสมกับการตรวจสอบว่าแสตมป์เป็นของจริงหรือปลอม ขั้นตอนของวิธีการวิจัยที่นำมาใช้มีดังนี้
\begin{enumerate}
\item ศึกษาข้อมูลเกี่ยวกับแสตมป์สุราและวิธีการตรวจสอบโดยการสอบถามและสนทนากับผู้เชี่ยวชาญการตรวจสอบ
\item ศึกษางานวิจัยในอดีตที่เกี่ยวข้องกับปัญหาวิจัย
\item ถ่ายภาพแสตมป์ตัวอย่างทั้งแสตมป์จริงและแสตมป์ปลอม
\item ค้นหาวิธีการประมวลผลภาพที่เหมาะสมบนฐานของการทดลอง
\item ทดสอบวิธีการที่ได้กับภาพแสตมป์ตัวอย่าง
\item วิจารณ์และสรุปผลการวิจัย

\end{enumerate}
