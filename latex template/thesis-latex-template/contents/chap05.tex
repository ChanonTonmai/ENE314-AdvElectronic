\chapter{บทสรุป}
โครงงานศึกษาวิจัยนี้ได้นำเสนอวิธีการตรวจสอบแสตมป์สุราว่าเป็นแสตมป์ของจริงหรือของปลอม โดยแสตมป์สุราที่สนใจเป็นแสตมป์สุราที่นำเข้า หรือเรียกอีกอย่างว่าแสตมป์สุราต่างประเทศซึ่งเป็นสุราที่มีราคาสูง ทำให้มีการลักลอบนำเข้า แล้วนำมาติดตราแสตมป์ปลอมเพื่อหลีกเลี่ยงการเสียภาษีสรรพากร 
วิธีที่กรมสรรพสามิตในเขตพื้นที่ต่าง ๆ ใช้ในการตรวจสอบแสตมป์สุรานำเข้าคือ การใช้ผู้เชี่ยวชาญในการตรวจ ซึ่งมีจำนวนคนน้อย จึงทำให้ไม่สามารถตรวจสอบการลักลอบได้อย่างทั่วถึง ปริมาณการลักลอบก็สูงขึ้นเนื่องจากผู้ลักลอบทราบว่าเจ้าหน้าที่มีน้อยจึงกล้าเสี่ยง

โครงงานศึกษาวิจัยนี้ได้เสนอให้ใช้การประมวลผลภาพในการตรวจสอบความเป็นของแท้ของแสตมป์สุรา โดยภาพที่ใช้ในการทดสอบเป็นภาพของแสตมป์ที่ถ่ายจากด้านบนของขวดด้วยกล้องดิจิทัล วิธีที่นำเสนอเป็นวิธีที่ง่ายโดยใช้ฮัพวงกลมในการตัดเอาเฉพาะตรานกวายุภักษ์ออกมาก่อน แล้วนำไปหาคุณลักษณะเด่น ก่อนที่จะนำไปตัดสินว่าเป็นแสตมป์จริงหรือปลอม คุณลักษณะเด่นที่ใช้ในการแยกเป็นคุณลักษณะของฮิสโตแกรมของสีเขียวของภาพนกภายุภักษ์ที่ตัดมา โดยมี 2 ตัวคือ ความสูงที่มากที่สุดของฮิสโตแกรมเรียกว่า $h$ และความกว้างของฮิสโตแกรมเรียกว่า $w$ วิธีการแยกแยะใช้เส้นตรงในระนาบของ $(w,h)$ ที่ได้จากการเรียนรู้แบบง่ายจากเซตของภาพแสตมป์ทั้งของจริงและของปลอม

ผลการทดสอบกับแสตมป์ที่ได้มาจากกรมสรรพสามิตเขตพื้นที่ จังหวัดอุบลราชธานี จำนวน 40 แสตมป์ เป็นแสตมป์จริงและปลอมอย่างละ 20 แสตมป์พบว่าวิธีที่นำเสนอสามารถตรวจสอบได้อย่างถูกต้องทั้งหมด แต่มีแสตมป์จริงบางตัวที่มีคุณสมบัติใกล้เคียงกับแส้นแบ่างที่ได้จากการเรียนรู้

อย่างไรก็ตาม แม้ว่าผลการทดสอบจะมีอัตราการตรวจสอบถูกต้องที่สูงเป็นที่น่าพอใจ แต่การที่จะนำวิธีการเช่นนี้ไปใช้จริงอาจจะต้องพิจารณาประเด็นต่าง ๆ อีกหลายประเด็น ได้แก่  (1) การถ่ายภาพที่จะต้องอยู่ในระยะทีกำหนด และอาจจะต้องควบคุมแสงให้เพียงพอ (2) ปรับปรุงวิธีการแยกแยะเช่นการใช้ SVM (support vector machine) แทนเพราะวิธีที่นำเสนอใช้หลักการเดียวกัน (3) มีการเรียนรู้ด้วยภาพที่มากกว่า 5 ภาพที่ใช้อยู่ และ (4) วิธีที่นำเสนออาจใช้ได้ดีกับการปลอมแปลงที่สีของตรานกวายุภักษ์มีความแตกต่างกัน แต่ถ้ามีการปลอมแปลงแบบอื่นวิธีนี้อาจจะใช้ไม่ได้ อย่างไรก็ตามหลักการของการตัดเอามาเฉพาะตรานกวายุภักษ์ซึ่งเป็นส่วนที่มีรายละเอียดมาก จึงยากที่สุดที่จะปลอมแปลง เพีื่อมาวิเคราะห์ว่าเป็นของแท้หรือของปลอมเป็นแนวคิดที่มีโอกาสที่จะใช้ได้สูง


