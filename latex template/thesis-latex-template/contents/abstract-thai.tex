 \thaimainfont
\thaiabstract
โครงงานศึกษาวิจัยนี้เสนอวิธีการตรวจสอบแสตมป์ภาษีสุรานำเข้าจากต่างประเทศว่าเป็นแสตมป์จริงหรือปลอมวิธีที่นำเสนอใช้การแปลงฮัฟมาตรฐานในการตัดเอานกวายุภักษ์ในแสตมป์ซึ่งเป็นส่วนที่สำคัญที่สุดออกมาและใช้ความกว้างและความสูงของอิสโตแกรมของภาพนกวายุภักษ์ที่ตัดมาเป็นคุณลักษณะในการตัดสินว่าแสตมป์เป็นของจริงหรือไม่ โดยที่ผ่านมาวิธีการตรวจสอบแสตมป์จะใช้เจ้าหน้าที่ผู้ชำนาญและมีประสบการณ์ทำการตรวจสอบและจำแนกแสตมป์แท้และแสตมป์สุราปลอมด้วยการดูด้วยสายตา ในขั้นตอนของการตัดเอาภาพนกวายุภักษ์ ภาพของแสตมป์ที่ถ่ายด้วยกล้องถ่ายภาพดิจิทัล จะถูกเปลี่ยนภาพระดับสีเทาก่อนที่จะถูกเปลี่ยนเป็นภาพขาวดำ เพื่อที่จะหาขอบภาพด้วยเทคนิคการหาขอบแบบแคนนี่ แล้วจึงใช้การแปลงฮัฟแบบมาตรฐานในการหาจุดศูนย์กลางและรัศมีของวงกลมสำหรับใช้ในการตัดเอาส่วนของรูปนกวายุภักษ์ แล้วหาความกว้างและความสูงของฮิสโตแกรมของรูปนกวายุภักษ์เพื่อใช้เป็นจุดพิกัดในการเทียบกับเส้นเกณฑ์ในการตัดสินว่าแสตมป์เป็นของจริงหรือไม่ เส้นเกณฑ์ดังกล่าวได้มาจากการเรียนรู้โดยใช้เซตของภาพตัวอย่าง ผลการทดสอบวิธีที่นำเสนอกับภาพแสตมป์ตัวอย่างจำนวน  40 ภาพเป็นภาพแสตมป์จริง 20 ภาพ ให้ผลการตรวจสอบถูกต้องทั้งหมด

%%%%%%%%%%%%%%%%
%Key words
%----------------------------------------
\begin{flushleft}
\begin{tabular*}{\textwidth}{@{}lp{0.8\textwidth}}
 & \\
คำสำคัญ: & วงกลมฮัฟ/ แสตมป์ภาษี/ ฮิสโตแกรม 
\\
\end{tabular*}
\end{flushleft}
\normalfont